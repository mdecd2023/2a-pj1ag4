\renewcommand{\baselinestretch}{1.5} %設定行距
\pagenumbering{roman} %設定頁數為羅馬數字
\clearpage  %設定頁數開始編譯
\sectionef
\addcontentsline{toc}{chapter}{摘~~~要} %將摘要加入目錄
\begin{center}
\LARGE\textbf{摘~~要}\\
\end{center}
\begin{flushleft}
\fontsize{14pt}{20pt}\sectionef\hspace{12pt}\quad 由於矩陣計算、自動求導技術、開源開發環境、多核GPU運算硬體等這四大發展趨勢,促使AI領域快速發展,藉由這樣的契機,將實體機電系統透過虛擬化訓練提高訓練效率,再將訓練完的模型應用到實體上。\\[12pt]

\fontsize{14pt}{20pt}\sectionef\hspace{12pt}\quad 此專案是運用w3作業所做的泡泡機器人,將其導入CoppeliaSim模擬環境並給予對應設置,將其機電系統簡化並導入Lua及python程式,再到CoppeliaSim模擬環境中進行測試演算法在實際運用上的可行性。並嘗試透過架設伺服器將CoppeliaSim影像串流到網頁供使用者觀看或操控。\\[12pt]

\end{flushleft}
\newpage
%=--------------------Abstract----------------------=%
\renewcommand{\baselinestretch}{1.5} %設定行距
\addcontentsline{toc}{chapter}{Abstract} %將摘要加入目錄
\begin{center}
\LARGE\textbf\sectionef{Abstract}\\
\begin{flushleft}
\fontsize{14pt}{16pt}\sectionef\hspace{12pt}\quad Due to the four major development trends of multidimensional arrays  computing, automatic differentiation, open source development environment, and multi-core GPUs computing hardware. The rapid development of the AI field has been promoted. In view of this development, the physical mechatronic systems can gain machine learning efficiency through their simulated virtual system training process. And afterwards to apply the trained model into real mechatronic systems.\\[12pt]

\fontsize{14pt}{16pt}\sectionef\hspace{12pt}\quad This project involves taking the bubbleRob created for the W3 assignment and importing it into the CoppeliaSim simulation environment with corresponding settings, simplifying its electromechanical system, and incorporating Lua and Python programming for testing algorithms in the CoppeliaSim simulation environment to determine their feasibility in practical application. Additionally, the project aims to stream CoppeliaSim images to a webpage for users to view or manipulate by setting up a server.\\
\end{flushleft}